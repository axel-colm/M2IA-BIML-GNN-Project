\section{Methode de classification}
Pour cette étude, j'ai choisi d'utiliser un modèle de Graph Convolutional Network (GCN) pour la classification des aéroports. 

\subsection{Graphe Convolutional Network (GCN)} 
Les GCN sont des réseaux de neurones spécifiquement conçus pour traiter des données structurées sous forme de graphes,  ce qui les rend particulièrement adaptés à notre problématique.  
Ils permettent d'apprendre des représentations vectorielles des nœuds (dans notre cas, les aéroports) en tenant compte à la fois des informations propres à chaque nœud (attributs) et de la structure du graphe (connexions avec les autres nœuds).\\
Ils sera donc intéressant d'interpréter les résultats obtenus pour comprendre comment les diffétents attributs et la structure du graphe influent sur sa capacité à classifier les aéroports par pays.\\
\\
Le modèle GCN que nous avons utilisé est composé de trois couches de convolution, chacune suivie d'une fonction d'activation ReLU. La dernière couche est suivie d'une fonction d'activation softmax
pour obtenir une distribution de probabilité sur les pays.\\
% Code embelling
\begin{lstlisting}[language=Python]
class GCN(nn.Module):
    def __init__(self, dim_in, dim_h, dim_out):
        super(GCN, self).__init__()
        self.conv1 = gnn.GCNConv(dim_in, dim_h)
        self.conv2 = gnn.GCNConv(dim_h, dim_h)
        self.conv3 = gnn.GCNConv(dim_h, dim_out)
        
        
    def forward(self, x, edge_index, edge_weight):
        x = F.relu(self.conv1(x, edge_index, edge_weight))
        x = F.relu(self.conv2(x, edge_index, edge_weight))
        x = self.conv3(x, edge_index, edge_weight)
        return  F.log_softmax(x, dim=1)
\end{lstlisting}


\subsection{Métriques d'évaluation}
Pour évaluer la performance de notre modèle, nous avons utilisé deux métriques d'évaluation: la précision. 
La précision est le nombre de prédictions correctes divisé par le nombre total de prédictions. Elle permet de mesurer la capacité du modèle à classifier correctement les aéroports par pays.\\



