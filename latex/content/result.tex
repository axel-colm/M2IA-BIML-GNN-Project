\section{Experimentation et résultats}

\subsection{Configurations testées}

Afin d'évaluer l'influence des différents facteurs sur la performance de la
classification des aéroports par pays, j'ai réalisé une série
d'expérimentations en faisant varier les paramètres suivants :

\begin{enumerate}
    \item {
          \textbf{Nombre d'aéroports par pays} : Permet de tester les performances du modèle en faisant varier le nombre d'aéroports par pays. Voici les configurations suivantes : 10, 20, 30, 40, 50, 60, 70, 80, 90 et 100 aéroports par pays.
          }
    \item {
          \textbf{Influence des features} : J'ai testé les performances du modèle en faisant varier les features utilisées pour la classification. Nous avons testé les configurations suivantes :[latitude, longitude]; [latitude, longitude, population]; [population].\\
          \textit{Note} : J'ai également fait varier le nombre d'aéroports par pays pour chaque configuration de features (10, 50, 100).
          }
    \item {
          \textbf{Pondération des liens} : J'ai testé sur différentes configurations de nombre d'aéroports par pays (10, 50, 100) l'influence de la pondération des liens sur la performance du modèle. Et comparé les résultats obtenus avec et sans pondération des liens.
          }
    \item {
          \textbf{Structure du graphe} : J'ai terminé par tester les performances du modèle en faisant varier la structure du graphe (Fully-connected, Country-based).
          }
\end{enumerate}

\subsection{Analyse des résultats}

Dans cette section, je vais analyser les résultats obtenus. Je m'appuierai
principalement sur les données de précision et d'erreur pour chaque
configuration testée.\\

\needspace{10\baselineskip}
\subsubsection{Nombre d'aéroports par pays}
Mon objectif ici à été de comprendre l'influence du nombre d'aéroports par pays
sur la performance du modèle.\\ J'ai donc mis en place un système de filtrage
avant de lancer l'entraînement du modèle.\\ Ce filtrage consiste à ne garder
que les pays ayant un nombre d'aéroports supérieur ou égal à la valeur de
l'hyperparamètre \textit{min\_airports\_per\_country}.\\ J'ai testé les valeurs
suivantes pour cet hyperparamètre : 1, 10, 20, 30, 40, 50, 60, 70, 80, 90 et
100.\\
\begin{table}[h!]
    \centering
    \begin{tabular}{|c|c||c|c|}
        \hline
        \textbf{Nombre d'aéroports par pays} & \textbf{Nombre de classes} & \textbf{Précision} & \textbf{Erreur} \\ \hline
        1                                    & 212                        & 0.6142             & 1.9571          \\ \hline
        10                                   & 67                         & 0.7458             & 1.199           \\ \hline
        20                                   & 41                         & 0.8282             & 1.024           \\ \hline
        30                                   & 24                         & 0.8838             & 0.6034          \\ \hline
        40                                   & 17                         & 0.8864             & 0.5404          \\ \hline
        50                                   & 12                         & 0.9379             & 0.306           \\ \hline
        60                                   & 8                          & 0.9805             & 0.0911          \\ \hline
        70                                   & 7                          & 0.9658             & 0.1553          \\ \hline
        80                                   & 6                          & 0.964              & 0.1197          \\ \hline
        90                                   & 6                          & 0.9928             & 0.0958          \\ \hline
        100                                  & 4                          & 0.9504             & 0.1045          \\ \hline
    \end{tabular}
    \caption{Résultat de la classification des aéroports par pays en fonction du nombre d'aéroports par pays}
\end{table}

\begin{figure}[h!]
    \centering
    \includegraphics[width=\textwidth]{content/img/matrix-f10.png}
    \caption{Matrice de confusion pour 10 aéroports par pays minimum}
    \label{fig:matrix-f10}
\end{figure}

\begin{figure}[h!]
    \centering
    \includegraphics[width=0.3\textwidth]{content/img/matrix-f90.png}
    \caption{Matrice de confusion pour 90 aéroports par pays minimum}
    \label{fig:matrix-f90}
\end{figure}

Les résultats obtenus montrent que, globalement (il y a des exceptions), la
performance du modèle augmente à mesure que l'on diminue le nombre de pays à
classifier.\\ Cela s'explique par le fait que plus le nombre d'aéroports par
pays est faible, plus il y aura de classes différentes à prédir, et plus il y
aura de chances de tomber sur un aéroport unique au pays. Le modèle devra donc
classifier un aéroport qui n'a encore jamais été vu.\\ Cela permet donc
d'obtenir une meilleure performance du modèle.\\

La matrices (Figure \ref{fig:matrix-f10}) montre que même avec un filtrage de
10 aéroports minimum par pays, le modèle a encore du mal à classifier certains
aéroports (certains pays ont une précision de 0\%). On peut la comparer à la
matrice (Figure \ref{fig:matrix-f90}) qui montre une meilleure performance du
modèle avec un filtrage de 90 aéroports minimum par pays. On peux d'ailleurs
observer qu'il y a quelques erreurs de classification pour les pays étant
proches géographiquement.\\

La nombre d'aéroports par pays est donc un facteur important à prendre en
compte pour la classification des aéroports par pays. Des classes (pays) avec
un nombre d'aéroports trop faible à pour effet d'introduire du bruit dans les
données et de diminuer la performance du modèle.\\

\needspace{20\baselineskip}
\subsubsection{Influence des features}

Dans cette partie, j'ai testé l'influence des features sur la performance du
modèle.\\ J'ai testé les configurations suivantes : [latitude, longitude];
[latitude, longitude, population]; [population].\\ J'ai également fait varier
le nombre d'aéroports par pays pour chaque configuration de features (10, 50,
100) car il pourrait y avoir un attribut qui influence plus ou moins la
performance du modèle en fonction du nombre de classes à prédire.\\

% Table
\begin{table}[h!]
    \centering

    \begin{tabular}{|c||c|c|c|c|c|c|}
        \hline
        \textbf{Nombre d'aéroports par pays} & \multicolumn{2}{c|}{10} & \multicolumn{2}{c|}{50} & \multicolumn{2}{c|}{100}                                                          \\ \hline
        \textbf{Nombre de classes}           & \multicolumn{2}{c|}{67} & \multicolumn{2}{c|}{12} & \multicolumn{2}{c|}{4}                                                            \\ \hline \hline
        \textbf{Features}                    & \textbf{Précision}      & \textbf{Erreur}         & \textbf{Précision}       & \textbf{Erreur} & \textbf{Précision} & \textbf{Erreur} \\ \hline
        [latitude, longitude]                & 0.6407                  & 1.5146                  & 0.9266                   & 0.2872          & 0.9587             & 0.1284          \\ \hline
        [latitude, longitude, population]    & 0.6712                  & 1.6314                  & 0.9492                   & 0.2192          & 0.9835             & 0.056           \\ \hline
        [population]                         & 0.1051                  & 3.5076                  & 0.1751                   & 1.7845          & 0.7769             & 0.5597          \\ \hline

    \end{tabular}
    \caption{Influence des features sur la performance du modèle}
\end{table}

L'analyse de 'influence des features sur la performance du modèle GCN révèle
des tendances intéressantes. Globalement, l'ajout de la population aux features
de latitude et longitude améliore la précision du modèle et réduit l'erreur de
classification.\\

La combinaison des trois features - latitude, longitude et population - s'avère
la plus efficace pour la classification des pays. Ceci suggère que la
population joue un rôle significatif dans la structuration du réseau aérien et
permet au modèle de mieux discriminer les pays. Par exemple, pour 10 aéroports
par pays, la précision passe de 0.6407 à 0.6712 lorsqu'on ajoute la population
aux features de latitude et longitude.\\

Cette constatation m'a amené à explorer davantage l'influence de la population
en l'isolant des autres features. Les résultats montrent que la population
seule n'est pas suffisante pour classifier un grand nombre de pays mais quand
le nombre de classes diminue, la population devient un facteur déterminant pour
la classification. Par exemple, pour 100 aéroports par pays, la précision est
de 0.7769 avec la population seule contre. La population doit donc avoir un
impact pour créer pluste de contraste entre les différents pays. On peut donc
en déduire que la différence de population entre les USA et le Canada
(adjacents géographiquement) est une information qui a été extraite par le
modèle pour les différencier.\\

\needspace{20\baselineskip}
\subsubsection{Pondération des liens}

Après avoir examiner les features des noeuds, j'ai décidé de m'intéresser à la
pondération des liens.\\ J'ai voulu savoir si la pondération des liens pouvait
améliorer la performance du modèle.\\ J'ai testé sur différentes configurations
de nombre d'aéroports par pays (10, 50, 100) l'influence de la pondération des
liens sur la performance du modèle. Et comparé les résultats obtenus avec et
sans pondération des liens.\\

Pour pondérer les liens j'ai calculé la distance euclidienne entre les
aéroports (grâce à leurs coordonnées latitude et longitude), je l'ai normalisé
et utiliser l'inverse de cette distance comme pondération (plus les aéroports
sont proches, plus le lien est fort).\\

% Table
\begin{table}[h!]
    \centering

    \begin{tabular}{|c||c|c|c|c|c|c|}
        \hline
        \textbf{Nombre d'aéroports par pays} & \multicolumn{2}{c|}{10} & \multicolumn{2}{c|}{50} & \multicolumn{2}{c|}{100}                                                          \\ \hline \hline
        \textbf{Pondération des liens}       & \textbf{Précision}      & \textbf{Erreur}         & \textbf{Précision}       & \textbf{Erreur} & \textbf{Précision} & \textbf{Erreur} \\ \hline
        Sans pondération                     & 0.6407                  & 1.5146                  & 0.9266                   & 0.2872          & 0.9587             & 0.1284          \\ \hline
        Avec pondération                     & 0.7017                  & 1.1801                  & 0.9379                   & 0.2429          & 0.9752             & 0.0822          \\ \hline

    \end{tabular}
    \caption{Influence de la pondération des liens sur la performance du modèle}
\end{table}

Les résultats montrent que la pondération des liens améliore la performance du
modèle.\\

L'ajout de la pondération aux liens a un effet positif majeur sur la précision
du modèle et sur la réduction de l'erreur, et ce, quel que soit le nombre
d'aéroports par pays. Par exemple, pour 10 aéroports par pays, la précision
passe de 0.6407 sans pondération à 0.9379 avec pondération, tandis que l'erreur
diminue de 1.5146 à 1.1801. Cette amélioration est observée de manière
consistante pour 50 et 100 aéroports par pays.\\

Cepandant il

\needspace{20\baselineskip}
\subsubsection{Structure du graphe}

Enfin, j'ai testé les performances du modèle en faisant varier la structure du
graphe (Fully-connected, Country-based).\\ J'ai testé les performances du
modèle avec un graphe \textbf{fully-connected} et un graphe
\textbf{country-based}.\\

Cela n'a pas pour but de trouver les meilleurs hyperparamètres mais d'observer
l'influence des liens entre les noeuds sur la performance d'un modèle GCN.\\

% Table fully-connected
\begin{table}[h!]
    \centering

    \begin{tabular}{|c||c|c|c|c|c|c|}
        \hline
        \textbf{Nombre d'aéroports par pays} & \multicolumn{2}{c|}{10} & \multicolumn{2}{c|}{50} & \multicolumn{2}{c|}{100}                                                          \\ \hline \hline
        \textbf{Structure du graph}          & \textbf{Précision}      & \textbf{Erreur}         & \textbf{Précision}       & \textbf{Erreur} & \textbf{Précision} & \textbf{Erreur} \\ \hline
        Fully-connected                      & /////                   & /////                   & /////                    & /////           & 0.5537             & 1.1546          \\ \hline
        Country-based                        & 0.9928                  & 0.0826                  & 1.0                      & 0.0043          & 1.                 & 0.0011          \\ \hline
        Original                             & 0.6407                  & 1.5146                  & 0.9266                   & 0.2872          & 0.9752             & 0.0822          \\ \hline
    \end{tabular}
    \caption{Performance du modèle avec un graphe \textbf{fully-connected}} avec pondération
\end{table}

\textit{Pour des raisons de performance, j'ai testé les graphes fully-connected avec pondération uniquement pour les pays ayant un minimum de 100 aéroports}.

Le graphe "country-based", où les aéroports sont connectés uniquement aux
aéroports du même pays, offre les meilleures performances en termes de
précision et d'erreur. Il atteint une précision parfaite (1.0) pour 50 et 100
aéroports par pays minimum, avec une erreur quasi nulle. Ceci suggère que la
structure "country-based" capture efficacement les relations essentielles pour
la classification des pays.\\

Le graphe "fully-connected", où tous les aéroports sont connectés entre eux,
présente la plus faible performance, avec une précision de seulement 0.5537
pour 100 aéroports par pays. Ceci indique qu'une connectivité excessive
introduit du bruit et nuit à la capacité du modèle à discriminer les pays.\\

Ces résultats soulignent l'importance de choisir une structure de graphe
adaptée à la tâche de classification. Dans votre cas, la structure
"country-based" semble la plus pertinente car elle reflète directement les
relations géographiques et politiques qui déterminent l'appartenance d'un
aéroport à un pays mais est une structure qui peut être difficile à obtenir
dans la réalité.\\

\subsection{Mon modèle de classification}

Après avoir testé différentes configurations, j'ai choisi un modèle de
classification qui combine les éléments suivants : \\ - \textbf{Features} :
[latitude, longitude, population] \\ - \textbf{Nombre d'aéroports par pays} :
30 \\ - \textbf{Pondération des liens} : Oui \\

J'ai choisi un nombre d'aéroports par pays de 30 car il offre un bon compromis
entre le nombre de classes à prédire et la performance du modèle.\\

\begin{table}[h!]
    \centering
    \begin{tabular}{|c|c|c|c|}
        \hline
                           & \textbf{Train} & \textbf{Validation} & \textbf{Test} \\ \hline \hline
        \textbf{Précision} & 0.95           & 0.9136              & 0.8909        \\ \hline
        \textbf{Erreur}    & 0.2018         & 0.3869              & 0.5313        \\ \hline
    \end{tabular}
    \caption{Performance du modèle final}
\end{table}

Le modèle final a été entraîné sur un total de 3 153 époches, avec un taux d'apprentissage de 0.001.
Le modèle a atteint une précision d'environ 90\%.\\

La matrice de confusion (Figure \ref{fig:matrix-f30}) montre que le modèle a
une bonne capacité à discriminer les pays, avec des taux de classification
élevés pour la plupart des classes. Cependant, il y a quelques erreurs de
classification pour certains pays, en particulier pour les pays voisins ou
géographiquement proches.\\
\begin{figure}[h!]
    \centering
    \includegraphics[width=0.4\textwidth]{content/img/final-matrix.png}
    \caption{Matrice de confusion du modèle final}
    \label{fig:matrix-f30}
\end{figure}


Nous pouvons également observer l'évolution de la précision et de l'erreur du
modèle par époque (Figures \ref{fig:acc-evol} \& \ref{fig:loss-evol}).\\

% image 
\begin{figure}[h!]
    \centering
    \includegraphics[width=\textwidth]{content/img/final-acc-evol.png}
    \caption{Evolution de la précision du modèle par époque}
    \label{fig:acc-evol}
\end{figure}

\begin{figure}[h!]
    \centering
    \includegraphics[width=\textwidth]{content/img/final-loss-evol.png}
    \caption{Evolution de l'erreur du modèle par époque}
    \label{fig:loss-evol}
\end{figure}


\clearpage  

\section{Conclusion}

Dans ce rapport, j'ai présenté une approche de classification des aéroports par
pays en utilisant un modèle GCN. J'ai exploré l'influence de différents facteurs
sur la performance du modèle, notamment le nombre d'aéroports par pays, les
features des noeuds, la pondération des liens et la structure du graphe.\\

J'ai constaté que le nombre d'aéroports par pays, les features des noeuds et la
pondération des liens ont un impact significatif sur la performance du modèle.
En particulier, l'ajout de la population aux features de latitude et longitude
améliore la précision du modèle, tandis que la pondération des liens renforce
les relations entre les aéroports et améliore la capacité du modèle à
discriminer les pays.\\

Enfin, j'ai identifié un modèle de classification optimal qui combine les
éléments suivants : features [latitude, longitude, population], 30 aéroports par
pays et pondération des liens. Ce modèle a atteint une précision d'environ 90\%
et est capable de prédire correctement le pays d'un aéroport dans la plupart des
cas.\\

En conclusion, cette approche de classification des aéroports par pays montre
que les modèles GCN peuvent être efficaces pour traiter des données complexes
et hétérogènes mais rencontre des limites quand le nombre de classes devient trop important.\\

Pour aller plus loin, il serait intéressant d'explorer d'autres types de modèles comme les GNNs récurrents ou les GNNs spatiaux pour améliorer la performance du modèle.

\begin{figure}[h!]
    \centering
    \includegraphics[width=\textwidth]{content/img/final-map.png}
    \caption{Prédictions du modèle final sur une carte}
    \label{fig:map}
\end{figure}
